%%%%%%%%%%%%%%%%%%%%%%%%%%%%%%%%%%%%%%%%%
% Developer CV
% LaTeX Template
% Version 1.0 (28/1/19)
%
% This template originates from:
% http://www.LaTeXTemplates.com
%
% Authors:
% Jan Vorisek (jan@vorisek.me)
% Based on a template by Jan Küster (info@jankuester.com)
% Modified for LaTeX Templates by Vel (vel@LaTeXTemplates.com)
%
% License:
% The MIT License (see included LICENSE file)
%
%%%%%%%%%%%%%%%%%%%%%%%%%%%%%%%%%%%%%%%%%

%----------------------------------------------------------------------------------------
%	PACKAGES AND OTHER DOCUMENT CONFIGURATIONS
%----------------------------------------------------------------------------------------

\documentclass[9pt]{developercv} % Default font size, values from 8-12pt are recommended

%----------------------------------------------------------------------------------------

\begin{document}

%----------------------------------------------------------------------------------------
%	TITLE AND CONTACT INFORMATION
%----------------------------------------------------------------------------------------

\begin{minipage}[t]{0.45\textwidth} % 45% of the page width for name
	\vspace{-\baselineskip} % Required for vertically aligning minipages
	
	% If your name is very short, use just one of the lines below
	% If your name is very long, reduce the font size or make the minipage wider and reduce the others proportionately
	\colorbox{black}{{\HUGE\textcolor{white}{\textbf{\MakeUppercase{Tomislav}}}}} % First name
	
	\colorbox{black}{{\HUGE\textcolor{white}{\textbf{\MakeUppercase{Nikic}}}}} % Last name
	
	\vspace{6pt}
	
	{\huge Software Engineer} % Career or current job title
\end{minipage}
\begin{minipage}[t]{0.275\textwidth} % 27.5% of the page width for the first row of icons
	\vspace{-\baselineskip} % Required for vertically aligning minipages
	
	% The first parameter is the FontAwesome icon name, the second is the box size and the third is the text
	% Other icons can be found by referring to fontawesome.pdf (supplied with the template) and using the word after \fa in the command for the icon you want
	\icon{Phone}{12}{+43 660 125 340 1}\\
	\icon{At}{12}{\href{mailto:hello@nikic.dev}{hello@nikic.dev}}\\	
\end{minipage}
\begin{minipage}[t]{0.275\textwidth} % 27.5% of the page width for the second row of icons
	\vspace{-\baselineskip} % Required for vertically aligning minipages
	
	% The first parameter is the FontAwesome icon name, the second is the box size and the third is the text
	% Other icons can be found by referring to fontawesome.pdf (supplied with the template) and using the word after \fa in the command for the icon you want
	\icon{Github}{12}{\href{https://github.com/tom1slav}{github.com/tom1slav}}\\
	\icon{Twitter}{12}{\href{https://twitter.com/@tomislav\_nikic}{@tomislav\_nikic}}\\
\end{minipage}

\vspace{0.5cm}

%----------------------------------------------------------------------------------------
%	INTRODUCTION, SKILLS AND TECHNOLOGIES
%----------------------------------------------------------------------------------------

\cvsect{Who Am I?}

\begin{minipage}[t]{0.4\textwidth} % 40% of the page width for the introduction text
	\vspace{-\baselineskip} % Required for vertically aligning minipages
	
	I am a software engineer with a lot of experience in framework development for test automation. After working in an all open source team, I have found my love for it. Sharing and improving work together makes coding fun. I love to learn new things all the time. Recently, I started to get interested in Machine Learning and its uses. In general, I always try to stay positive in life and push my own knowledge.\\
\end{minipage}
\hfill % Whitespace between
\begin{minipage}[t]{0.5\textwidth} % 50% of the page for the skills bar chart
	\vspace{-\baselineskip} % Required for vertically aligning minipages
	\begin{barchart}{5.5}
		\baritem{Ruby}{80}
		\baritem{JavaScript}{60}
		\baritem{Git}{100}
		\baritem{Selenium}{90}
		\baritem{Linux}{80}
		\baritem{Machine Learning}{40}
	\end{barchart}
\end{minipage}

%----------------------------------------------------------------------------------------
%	EXPERIENCE
%----------------------------------------------------------------------------------------

\cvsect{Experience}

\begin{entrylist}
	\entry
		{2019 -- 2022}
		{Software Engineer in Test}
		{GitLab Inc.}
		{I was working as a software engineer in test for GitLab. My team and I were developing a framework using Selenium and Capybara for easier test implementation and execution.\\ \texttt{Ruby}\slashsep\texttt{Selenium}\slashsep\texttt{Pact.io}}
	\entry
		{2018 -- 2019}
		{Software Engineer, Test Automation}
		{Shpock GmbH}
		{I was responsible for implementing a new framework for mobile automated testing. Since the product was mainly used from the app, this was vital for the company.\\ \texttt{Node.js}\slashsep\texttt{Appium}\slashsep\texttt{WebdriverIO}}
	\entry
		{2017 -- 2018}
		{Test Automation Engineer}
		{Anecon GmbH}
		{This was my introduction to test automation as a concept. I worked on multiple sites. Due to this, I witnessed a plethora of different projects and automation designs.\\ \texttt{C\#}\slashsep\texttt{Selenium}\slashsep\texttt{SAP}}
\end{entrylist}

%----------------------------------------------------------------------------------------
%	EDUCATION
%----------------------------------------------------------------------------------------

\cvsect{Education}

\begin{entrylist}
	\entry
		{2016 -- 2018}
		{Master's Degree}
		{University of Applied Sciences, Technikum Wien}
		{I have a masters degree in software development. During my studies I worked on many small and medium sized projects. My master thesis was about quality automation.}
	\entry
		{2017}
		{ISTQB Foundation Level}
		{Austrian Testing Board}
		{During my first employment in the field of test automation, I got the opportunity to pass my ISTQB certificate. This is a fundamental software testing certificate.}
	\entry
		{2011 -- 2016}
		{Bachelor's Degree}
		{University of Vienna}
		{My bachelor studies were in the field of health informatics like imaging technologies. My thesis was in the field of medical data collection systems.}
\end{entrylist}

%----------------------------------------------------------------------------------------
%	ADDITIONAL INFORMATION
%----------------------------------------------------------------------------------------

\begin{minipage}[t]{0.3\textwidth}
	\vspace{-\baselineskip} % Required for vertically aligning minipages

	\cvsect{Languages}
	
	\textbf{German} - native\\
	\textbf{Croatian} - native\\
	\textbf{English} - proficient
\end{minipage}
\hfill
\begin{minipage}[t]{0.3\textwidth}
	\vspace{-\baselineskip} % Required for vertically aligning minipages
	
	\cvsect{Hobbies}
	
	I love working with wood and improving my own home with it. Also taking pictures of all the things.
\end{minipage}
\hfill
\begin{minipage}[t]{0.3\textwidth}
	\vspace{-\baselineskip} % Required for vertically aligning minipages
	
	\cvsect{Non profit}
	
	My family and I try to help kids in need. Due to war, a lot of children are alone in a new country.
\end{minipage}

%----------------------------------------------------------------------------------------

\end{document}
